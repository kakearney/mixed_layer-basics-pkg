% \newcolumntype{x}[1]{>{\raggedleft\hspace{0pt}}p{#1}}     % p, but right-justified 
% \newcolumntype{y}[1]{>{\raggedright\hspace{0pt}}p{#1}}    % p, but left-justified

\newcolumntype{L}[1]{>{\RaggedRight}p{#1}} % p, but left-justified
\newcolumntype{R}[1]{>{\RaggedLeft}p{#1}}  % p, but right-justified

% Biogeochemical parameters tables

\newcommand{\bgctable}[3]{
\begin{table}[h]
	\footnotesize
	\caption{Biogeochemical process-related parameters: #2}
	\begin{tabular}{p{2in}p{0.5in}p{0.75in}>{\raggedleft}p{0.5in}@{ }p{1.25in}}
		\toprule
	  	Parameter & Symbol & Group &  Value & \\ % \multicolumn{2}{l}{Value} \\
		\midrule
		#1
		\bottomrule
   	\end{tabular}
	\label{tab:nemparam#3}
\end{table}
}

% Description of critters table

\newcommand{\crittertable}[2]{
\begin{table}
	\footnotesize   
	\caption{A description of the 47 functional groups included in the 
	         unsimplified version of the food web model. Species listed in the 
	         Includes column are not exhaustive, but represent the dominant 
	         members of each functional group.}
	\renewcommand{\arraystretch}{1.5}
	\begin{tabular}{L{1.25in}L{2.0in}L{3.00in}}
		\toprule
		Group & Includes & Details \tabularnewline
		\midrule 
        #1
		\bottomrule 
	\end{tabular}
	\label{tab:aydin48critterlist#2}
\end{table}
}

\newcommand{\crittertablelong}[1]{
\footnotesize
\renewcommand{\arraystretch}{1.5}
\begin{longtable}{L{1.25in}L{2.0in}L{2.75in}}
	\caption{A description of the 47 functional groups included in the 
	         unsimplified version of the food web model. Species listed in the 
	         Includes column are not exhaustive, but represent the dominant 
	         members of each functional group.} \\
	\toprule
	Group & Includes & Details \\
	\midrule
	\endfirsthead
	\caption*{A description of the 47 functional groups included in the 
	         unsimplified version of the food web model (continued).} \\
	\toprule
	Group & Includes & Details \\
	\midrule
	\endhead
	\bottomrule
	\endfoot
    #1
	\label{tab:aydin48critterlist}
\end{longtable}
\renewcommand{\arraystretch}{1.0}
\normalsize
}


% Time-varying parameters

\newcommand{\timevarytable}[1]{
\begin{table}
	\footnotesize  
	\caption{Derived parameters. These parameters vary over time as a function 
	         of the state variables from both the physical and biological
	         models.}   
	\renewcommand{\arraystretch}{2.5} 
	\begin{tabular}{lll}
		\toprule
		Parameter Name & Symbol & Definition\tabularnewline 
		\midrule 
		#1
		\bottomrule 
	\end{tabular}
	\label{tab:wcenemDerivedParams}
\end{table}
}

% Ecopath basic input

\newcommand{\ecopathbasictable}[1]{
\begin{table}
	\footnotesize
	\caption{Ecopath basic input variables for the 33-group simplified food 
	         web, including biomass (B, tons wet weight m$^{-2}$), 
	         production/biomass (PB, yr$^{-1}$), consumption/biomass 
	         (QB, yr$^{-1}$), ecotrophic efficiency (EE), growth efficiency 
	         (GE), and fraction unassimilated (GS)}
	\begin{tabular}{L{1.50in}*{11}{r}}
		\toprule
	  	Group 	& \multicolumn{2}{c}{B}  &  \multicolumn{2}{c}{PB} & \multicolumn{2}{c}{QB} & \multicolumn{2}{c}{EE} & \multicolumn{2}{c}{GE}  & GS  \\     
	    \cmidrule(lr){2-3}\cmidrule(lr){4-5}\cmidrule(lr){6-7}\cmidrule(lr){8-9}\cmidrule(lr){10-11}
		        & Value     & Ped        & Value    & Ped          & Value    & Ped         & Value    & Ped         &  Value    & Ped         &     \\ 
		\midrule
		#1
		\bottomrule
   	\end{tabular}
	\label{tab:ecopathIn33}
\end{table}
}

% Ecopath diet input

\newcommand{\diettable}[2]{
\begin{table}
	\footnotesize 
	\caption{Ecopath diet fraction input for the 33-group food web model.}    
	\begin{tabular}{L{0.3\textwidth}L{0.3\textwidth}R{0.1\textwidth}R{0.1\textwidth}}
		\toprule
		Predator & Prey & Diet percentage & Pedigree \tabularnewline
		\midrule 
		#1
		\bottomrule 
	\end{tabular}
	\label{tab:diet33_#2}
\end{table}
}


\newcommand{\diettablelong}[1]{
\footnotesize 
\begin{longtable}[l]{L{0.3\textwidth}L{0.3\textwidth}R{0.1\textwidth}R{0.1\textwidth}}
	\caption{Ecopath diet fraction input for the 33-group food web model.} \\    
	\toprule
	Predator & Prey & \mbox{Diet percentage} & Pedigree \tabularnewline
	\midrule 
	\endfirsthead
	\caption*{Ecopath diet fraction input for the 33-group food web model (continued).} \\    
	\toprule
	Predator & Prey & \mbox{Diet percentage} & Pedigree \tabularnewline
	\midrule 
	\endhead
	\bottomrule 
	\endfoot
	#1
	\label{tab:diet33}
\end{longtable}
\normalsize
}

% Ecopath-derived, state-variable-related

\newcommand{\epderivedstatetable}[1]{
\begin{table}
    \footnotesize
    \caption{Ecopath-derived parameters for living state variables, including 
             mass-balanced biomass ($B^*$, mol N m$^{-2}$), mass-balanced mortality 
             flux per unit biomass ($M_0$, s$^{-1}$), growth efficiency ($GE$), 
             and unassimilation fraction ($GS$).  The histogram columns indicate 
             the distribution of values across ensemble members, with the column 
             width ranging from 0 to 3.5 times the mean value.}
    \begin{tabular}{llllllll}
        \toprule
        Group & \multicolumn{2}{c}{B*} & \multicolumn{2}{c}{M0} & \multicolumn{2}{c}{GE} & GS \\
        \cmidrule(r){2-3} \cmidrule(r){4-5} \cmidrule(r){6-7}
              & mean & histogram       & mean & histogram       & mean & histogram       & \\
        \midrule
		#1
        \bottomrule
    \end{tabular}
    \label{tab:ecopathderivedState}
\end{table}
}

% Ecopath-derived, predator-prey link-related

\newcommand{\epderivedlinktable}[2]{
\begin{table}
    \footnotesize
    \caption{Ecopath-derived parameters for predator-prey processes, including mass-balanced consumption rate ($Q^*$, mol N m$^{-2}$ s$^{-1}$), top-down control parameter ($X$), bottom-up control parameter ($D$), and functional response exponent ($\theta$).  The histogram column indicates the distribution of values across ensemble members, with the column width ranging from 0 to 8 times the mean value.}
    \begin{tabular}{L{0.25\textwidth}L{0.25\textwidth}llrrr}
        \toprule
        Predator & Prey & \multicolumn{2}{c}{Q*} & X & D & $\theta$ \\
        \cmidrule(r){3-4}
                 &      & mean & histogram       &   &   &          \\
        \midrule
		#1
        \bottomrule
    \end{tabular}
    \label{tab:ecopathderivedLink_#2}
\end{table}
}

\newcommand{\epderivedlinklong}[1]{
\footnotesize
\begin{longtable}[l]{L{0.28\textwidth}L{0.28\textwidth}llrrr}
	\caption{Ecopath-derived parameters for predator-prey processes, 
	         including mass-balanced consumption rate ($Q^*$, mol N 
	         m$^{-2}$ s$^{-1}$), top-down control parameter ($X$), 
	         bottom-up control parameter ($D$), and functional response 
	         exponent ($\theta$).  The histogram column indicates the 
	         distribution of values across ensemble members, with the 
	         column width ranging from 0 to 3.5 times the mean value.} \\
	\toprule
    Predator & Prey & \multicolumn{2}{c}{Q*} & X & D & $\theta$ \\
    \cmidrule(r){3-4}
             &      & mean & histogram       &   &   &          \\
    \midrule
	\endfirsthead
	\caption*{Ecopath-derived parameters for predator-prey processes (continued).} \\
	\toprule
    Predator & Prey & \multicolumn{2}{c}{Q*} & X & D & $\theta$ \\
    \cmidrule(r){3-4}
             &      & mean & histogram       &   &   &          \\
    \midrule
	\endhead
	\bottomrule
	\endfoot
	#1
	\label{tab:ecopathderivedLink}
\end{longtable}
\normalsize
}
		




